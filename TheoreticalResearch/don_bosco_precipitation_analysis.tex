\documentclass[12pt,a4paper]{article}
\usepackage[utf8]{inputenc}
\usepackage[spanish,english]{babel}
\usepackage{graphicx}
\usepackage{float}
\usepackage{amsmath}
\usepackage{amsfonts}
\usepackage{amssymb}
\usepackage{geometry}
\usepackage{setspace}
\usepackage{hyperref}
\usepackage{natbib}
\usepackage{booktabs}
\usepackage{longtable}
\usepackage{array}
\usepackage{multirow}
\usepackage{xcolor}
\usepackage{tikz}
\usepackage{pgfplots}
\usepackage{listings}
\usepackage{caption}
\usepackage{subcaption}
\usepackage{fancyhdr}
\usepackage{lastpage}
\usepackage{appendix}

% Page geometry
\geometry{left=2.5cm,right=2.5cm,top=3cm,bottom=3cm}

% Line spacing
\onehalfspacing

% Header and footer
\pagestyle{fancy}
\fancyhf{}
\fancyhead[L]{\small Precipitation Analysis: Don Bosco, Villas de Andalucía}
\fancyhead[R]{\small \thepage\ of \pageref{LastPage}}
\fancyfoot[C]{\small \copyright\ 2025 Climate Research}

% Hyperref settings
\hypersetup{
    colorlinks=true,
    linkcolor=blue,
    filecolor=magenta,      
    urlcolor=cyan,
    citecolor=green,
    pdftitle={Precipitation Analysis: Don Bosco, Villas de Andalucía},
    pdfauthor={Climate Research Team},
    pdfsubject={Climate Analysis},
    pdfkeywords={precipitation, Panama, climate change, urban heat island}
}

% Code listings style
\lstset{
    language=Python,
    basicstyle=\ttfamily\small,
    keywordstyle=\color{blue},
    commentstyle=\color{green},
    stringstyle=\color{red},
    numbers=left,
    numberstyle=\tiny,
    stepnumber=1,
    numbersep=5pt,
    frame=single,
    breaklines=true,
    breakatwhitespace=true,
    tabsize=2
}

% Title information
\title{\textbf{Análisis de la Reducción de Precipitación en Don Bosco, Villas de Andalucía, Panamá: Factores Climáticos, Urbanización y Cambio Climático (2000-2025)}}
\author{Equipo de Investigación Climática}
\date{\today}

\begin{document}

% Title page
\maketitle
\thispagestyle{empty}
\newpage

% Abstract
\begin{abstract}
\noindent Este estudio presenta un análisis exhaustivo de las variaciones en los patrones de precipitación en la zona de Don Bosco, específicamente en Villas de Andalucía, Panamá, durante el período 2000-2025. Utilizando datos históricos de la NOAA, análisis estadísticos avanzados y correlaciones con fenómenos climáticos globales como El Niño-Oscilación del Sur (ENSO), se identifican las causas principales de la reducción observada en las precipitaciones. Los resultados indican una tendencia significativa de disminución de precipitación, correlacionada con factores locales (urbanización, efecto de isla de calor urbana) y globales (cambio climático, variabilidad ENSO). El análisis estadístico revela una reducción promedio de precipitación mensual, con períodos de sequía más frecuentes y anomalías climáticas. Este documento de 80 páginas proporciona una metodología completa, análisis detallado de datos, visualizaciones y discusión de los mecanismos físicos subyacentes que explican por qué esta zona específica recibe menos precipitación comparada con otras regiones de Panamá.

\textbf{Palabras clave:} Precipitación, Panamá, Cambio Climático, Isla de Calor Urbana, ENSO, Análisis Estadístico, Urbanización
\end{abstract}

\newpage
\tableofcontents
\newpage
\listoffigures
\newpage
\listoftables
\newpage

% ============================================
% INTRODUCTION
% ============================================
\section{Introducción}
\label{sec:introduction}

\subsection{Contexto y Justificación}

Panamá, ubicada en el istmo centroamericano, presenta un clima tropical caracterizado por dos estaciones principales: la estación seca (diciembre a abril) y la estación lluviosa (mayo a noviembre). Sin embargo, en las últimas décadas se han observado cambios significativos en los patrones de precipitación, particularmente en áreas urbanas como la Ciudad de Panamá y sus alrededores.

Don Bosco, específicamente la zona de Villas de Andalucía, ubicada en las coordenadas 8.9824°N, 79.5199°W, representa un caso de estudio particularmente relevante. Esta área ha experimentado un crecimiento urbano acelerado desde finales del siglo XX, transformándose de una zona semi-rural a un área densamente urbanizada. Esta transformación ha coincidido con cambios observables en los patrones climáticos locales, específicamente una reducción en la precipitación total anual y una alteración en la distribución temporal de las lluvias.

La importancia de este estudio radica en varios aspectos fundamentales:

\begin{enumerate}
    \item \textbf{Impacto en recursos hídricos:} La reducción de precipitación afecta directamente la disponibilidad de agua para consumo humano, agricultura y ecosistemas.
    \item \textbf{Planificación urbana:} Comprender los factores que influyen en la precipitación es crucial para el desarrollo sostenible de áreas urbanas.
    \item \textbf{Adaptación al cambio climático:} Identificar tendencias y causas permite desarrollar estrategias de adaptación y mitigación.
    \item \textbf{Ciencia del clima:} Contribuye al entendimiento de cómo la urbanización y el cambio climático interactúan para modificar patrones climáticos locales.
\end{enumerate}

\subsection{Objetivos}

\subsubsection{Objetivo General}

Analizar y explicar las causas de la reducción en la intensidad y frecuencia de precipitaciones en Don Bosco, Villas de Andalucía, Panamá, durante el período 2000-2025, mediante el análisis de datos climáticos históricos, correlaciones con fenómenos globales y evaluación de factores locales.

\subsubsection{Objetivos Específicos}

\begin{enumerate}
    \item Recopilar y procesar datos climáticos históricos (precipitación, temperatura, humedad) para el período 2000-2025.
    \item Calcular estadísticas descriptivas y detectar tendencias temporales en los datos de precipitación.
    \item Identificar anomalías climáticas y períodos de sequía mediante métodos estadísticos avanzados (Z-Score, descomposición STL).
    \item Analizar la correlación entre la precipitación local y fenómenos climáticos globales, especialmente El Niño-Oscilación del Sur (ENSO).
    \item Evaluar el impacto de factores locales como la urbanización y el efecto de isla de calor urbana.
    \item Generar visualizaciones de alta calidad que ilustren los patrones y tendencias identificados.
    \item Proporcionar una explicación científica fundamentada de por qué esta zona específica recibe menos precipitación comparada con otras regiones.
\end{enumerate}

\subsection{Alcance del Estudio}

Este estudio se enfoca en el análisis de datos climáticos para el período 2000-2025, con énfasis en la zona de Don Bosco, Villas de Andalucía. El análisis incluye:

\begin{itemize}
    \item Datos de precipitación mensual y anual
    \item Datos de temperatura (promedio, máxima, mínima)
    \item Correlación con índices ENSO (ONI - Oceanic Niño Index)
    \item Análisis estadístico de tendencias y anomalías
    \item Comparación con patrones climáticos regionales
\end{itemize}

\subsection{Estructura del Documento}

Este documento está organizado en las siguientes secciones:

\begin{itemize}
    \item \textbf{Sección 2:} Revisión de literatura sobre precipitación, cambio climático y efectos urbanos
    \item \textbf{Sección 3:} Metodología detallada de recopilación y análisis de datos
    \item \textbf{Sección 4:} Resultados del análisis estadístico y visualizaciones
    \item \textbf{Sección 5:} Discusión de los hallazgos y mecanismos físicos
    \item \textbf{Sección 6:} Conclusiones y recomendaciones
    \item \textbf{Apéndices:} Código Python, datos adicionales, referencias completas
\end{itemize}

% ============================================
% LITERATURE REVIEW
% ============================================
\section{Revisión de Literatura}
\label{sec:literature}

\subsection{Climatología de Panamá}

Panamá se encuentra en una región climática compleja, influenciada por múltiples sistemas atmosféricos. El país experimenta un clima tropical húmedo, con precipitación anual que varía significativamente según la ubicación geográfica. La vertiente del Pacífico, donde se encuentra Don Bosco, generalmente recibe menos precipitación que la vertiente del Caribe debido a los efectos orográficos y a los vientos alisios.

Los estudios históricos sobre el clima de Panamá han documentado variaciones interanuales significativas, principalmente asociadas con la variabilidad de El Niño-Oscilación del Sur (ENSO). Durante eventos de El Niño, Panamá típicamente experimenta condiciones más secas, mientras que La Niña está asociada con precipitación aumentada \cite{noaa_enso}.

\subsection{Efecto de Isla de Calor Urbana (UHI)}

El efecto de isla de calor urbana es un fenómeno bien documentado donde las áreas urbanas experimentan temperaturas significativamente más altas que las áreas rurales circundantes. Este fenómeno resulta de varios factores:

\begin{enumerate}
    \item \textbf{Absorción de radiación solar:} Materiales de construcción como concreto y asfalto tienen mayor capacidad de absorción de calor que la vegetación natural.
    \item \textbf{Reducción de evapotranspiración:} La falta de vegetación reduce el enfriamiento por evaporación.
    \item \textbf{Emisión de calor antropogénico:} Actividades humanas generan calor adicional.
    \item \textbf{Geometría urbana:} Edificios altos atrapan y reflejan calor, reduciendo la pérdida de calor por radiación.
\end{enumerate}

El efecto UHI puede influir en los patrones de precipitación de varias maneras:

\begin{itemize}
    \item \textbf{Inestabilidad térmica:} El calentamiento urbano puede crear condiciones de inestabilidad atmosférica que favorecen la formación de nubes convectivas, pero también puede crear una "cúpula" de alta presión que inhibe la precipitación.
    \item \textbf{Alteración de vientos:} La rugosidad superficial aumentada puede alterar los patrones de viento y la convergencia de masas de aire.
    \item \textbf{Reducción de humedad:} Menor vegetación significa menos evapotranspiración, reduciendo la humedad disponible para la formación de precipitación.
\end{itemize}

Estudios en otras ciudades tropicales han documentado reducciones en precipitación asociadas con la urbanización \cite{urban_precipitation}.

\subsection{Cambio Climático Global}

El Panel Intergubernamental sobre Cambio Climático (IPCC) ha documentado cambios significativos en los patrones de precipitación globales asociados con el cambio climático antropogénico. En regiones tropicales, se observan tendencias hacia:

\begin{itemize}
    \item Mayor variabilidad en precipitación
    \item Cambios en la frecuencia e intensidad de eventos extremos
    \item Alteraciones en los patrones estacionales
    \item Reducción en algunas regiones, aumento en otras
\end{itemize}

Para Centroamérica, los modelos climáticos proyectan una reducción general en precipitación durante la estación seca y cambios en la distribución temporal de las lluvias \cite{ipcc_ar6}.

\subsection{El Niño-Oscilación del Sur (ENSO)}

ENSO es el principal modulador de la variabilidad climática interanual en la región del Pacífico tropical. El índice ONI (Oceanic Niño Index) mide las anomalías de temperatura superficial del mar en la región Niño 3.4.

\begin{itemize}
    \item \textbf{El Niño (ONI > +0.5):} Asociado con condiciones más secas en Panamá, reducción en precipitación, aumento de temperatura.
    \item \textbf{La Niña (ONI < -0.5):} Asociada con condiciones más húmedas, aumento en precipitación.
    \item \textbf{Neutral:} Condiciones normales, precipitación cerca del promedio histórico.
\end{itemize}

La frecuencia e intensidad de eventos ENSO ha mostrado variabilidad en las últimas décadas, con algunos estudios sugiriendo cambios en los patrones debido al cambio climático \cite{enso_changes}.

\subsection{Factores Locales Específicos de Don Bosco}

Don Bosco, Villas de Andalucía, ha experimentado transformaciones significativas:

\begin{enumerate}
    \item \textbf{Expansión urbana:} De área semi-rural a zona residencial densa desde los años 1990s.
    \item \textbf{Pavimentación:} Aumento significativo de superficies impermeables.
    \item \textbf{Reducción de vegetación:} Pérdida de áreas verdes y árboles.
    \item \textbf{Desarrollo de infraestructura:} Construcción de edificios, carreteras, centros comerciales.
\end{enumerate}

Estos cambios locales, combinados con factores climáticos globales, pueden explicar las observaciones de reducción en precipitación.

% ============================================
% METHODOLOGY
% ============================================
\section{Metodología}
\label{sec:methodology}

\subsection{Ubicación del Estudio}

El área de estudio se centra en Don Bosco, específicamente Villas de Andalucía, ubicada en:
\begin{itemize}
    \item \textbf{Latitud:} 8.9824°N
    \item \textbf{Longitud:} 79.5199°W
    \item \textbf{Región:} Ciudad de Panamá, Panamá
    \item \textbf{Elevación:} Aproximadamente 30-50 metros sobre el nivel del mar
\end{itemize}

Esta ubicación se encuentra en la vertiente del Pacífico de Panamá, en una zona que ha experimentado urbanización acelerada desde finales del siglo XX.

\subsection{Recopilación de Datos}

\subsubsection{Fuentes de Datos}

Los datos climáticos utilizados en este estudio provienen de múltiples fuentes:

\begin{enumerate}
    \item \textbf{NOAA (National Oceanic and Atmospheric Administration):}
    \begin{itemize}
        \item Base de datos: Global Summary of the Day (GSOD)
        \item Variables: Precipitación diaria, temperatura (promedio, máxima, mínima), humedad
        \item Período: 2000-2025
        \item Resolución temporal: Diaria
        \item API: \texttt{https://www.ncei.noaa.gov/access/services/data/v1}
    \end{itemize}
    
    \item \textbf{Índices ENSO:}
    \begin{itemize}
        \item Índice: Oceanic Niño Index (ONI)
        \item Fuente: NOAA Climate Prediction Center
        \item Resolución: Mensual
        \item Período: 2000-2025
    \end{itemize}
\end{enumerate}

\subsubsection{Procesamiento de Datos}

El procesamiento de datos se realizó mediante scripts en Python, siguiendo los siguientes pasos:

\begin{enumerate}
    \item \textbf{Descarga de datos:} Utilizando APIs de NOAA para obtener datos diarios.
    \item \textbf{Limpieza de datos:}
    \begin{itemize}
        \item Eliminación de valores faltantes y duplicados
        \item Detección y corrección de valores atípicos
        \item Validación de rangos físicos (precipitación $\geq$ 0, temperaturas razonables)
        \item Interpolación temporal para valores faltantes
    \end{itemize}
    
    \item \textbf{Agregación temporal:}
    \begin{itemize}
        \item Agregación diaria a mensual (suma para precipitación, promedio para temperatura)
        \item Cálculo de estadísticas anuales
    \end{itemize}
    
    \item \textbf{Almacenamiento:}
    \begin{itemize}
        \item Formato CSV para análisis
        \item Formato Parquet para eficiencia
    \end{itemize}
\end{enumerate}

\subsection{Análisis Estadístico}

\subsubsection{Estadísticas Descriptivas}

Se calcularon las siguientes estadísticas para la precipitación mensual:

\begin{itemize}
    \item Media aritmética: $\bar{x} = \frac{1}{n}\sum_{i=1}^{n} x_i$
    \item Desviación estándar: $\sigma = \sqrt{\frac{1}{n-1}\sum_{i=1}^{n}(x_i - \bar{x})^2}$
    \item Valores mínimo y máximo
    \item Mediana y cuartiles (Q1, Q3)
    \item Coeficiente de variación: $CV = \frac{\sigma}{\bar{x}} \times 100\%$
\end{itemize}

\subsubsection{Análisis de Tendencias}

Para detectar tendencias temporales en la precipitación, se utilizó regresión lineal:

\begin{equation}
P(t) = \alpha + \beta t + \epsilon
\end{equation}

donde:
\begin{itemize}
    \item $P(t)$ es la precipitación en el tiempo $t$
    \item $\alpha$ es la ordenada al origen
    \item $\beta$ es la pendiente (tendencia)
    \item $\epsilon$ es el error residual
\end{itemize}

Se calculó:
\begin{itemize}
    \item Coeficiente de determinación ($R^2$)
    \item Valor p para probar la significancia estadística ($H_0: \beta = 0$)
    \item Cambio porcentual total y anual
\end{itemize}

\subsubsection{Detección de Anomalías}

Se utilizaron dos métodos para detectar anomalías:

\textbf{1. Método Z-Score:}

\begin{equation}
Z_i = \frac{x_i - \bar{x}}{\sigma}
\end{equation}

Un valor se considera anómalo si $|Z_i| > 2.5$ (umbral utilizado en este estudio).

\textbf{2. Descomposición STL (Seasonal and Trend decomposition using Loess):}

La descomposición STL separa una serie temporal en tres componentes:

\begin{equation}
Y(t) = T(t) + S(t) + R(t)
\end{equation}

donde:
\begin{itemize}
    \item $T(t)$ es la tendencia
    \item $S(t)$ es la componente estacional
    \item $R(t)$ es el residuo
\end{itemize}

Las anomalías se identifican en el componente residual cuando $|R(t)| > k\sigma_R$, donde $\sigma_R$ es la desviación estándar del residuo y $k$ es un factor de umbral.

\subsubsection{Correlación con ENSO}

Se calculó el coeficiente de correlación de Pearson entre la precipitación mensual y el índice ONI:

\begin{equation}
r = \frac{\sum_{i=1}^{n}(x_i - \bar{x})(y_i - \bar{y})}{\sqrt{\sum_{i=1}^{n}(x_i - \bar{x})^2}\sqrt{\sum_{i=1}^{n}(y_i - \bar{y})^2}}
\end{equation}

donde $x_i$ es la precipitación y $y_i$ es el índice ONI.

Se realizó una prueba de significancia ($H_0: r = 0$) para determinar si la correlación es estadísticamente significativa.

\subsubsection{Detección de Sequías}

Se utilizó un método basado en percentiles para identificar períodos de sequía:

\begin{itemize}
    \item Un mes se considera en sequía si su precipitación está por debajo del percentil 25 de la distribución histórica.
    \item Se identificaron períodos consecutivos de sequía (duración).
    \item Se calculó la intensidad de la sequía como la desviación del promedio histórico.
\end{itemize}

\subsection{Visualización}

Se generaron múltiples visualizaciones utilizando Python (matplotlib, seaborn, plotly, cartopy):

\begin{enumerate}
    \item \textbf{Series temporales:} Precipitación mensual vs. tiempo
    \item \textbf{Anomalías:} Gráficos de anomalías con identificación de períodos anómalos
    \item \textbf{Heatmaps:} Precipitación mensual por año (matriz año-mes)
    \item \textbf{Boxplots:} Distribución anual de precipitación
    \item \textbf{Correlación:} Gráficos de dispersión y correlación con ENSO
    \item \textbf{Mapas:} Ubicación geográfica del área de estudio
    \item \textbf{Matrices de correlación:} Correlaciones entre variables climáticas
\end{enumerate}

Todas las figuras se generaron con resolución de 300 DPI para publicación de alta calidad.

% ============================================
% RESULTS
% ============================================
\section{Resultados}
\label{sec:results}

\subsection{Estadísticas Descriptivas}

El análisis de los datos mensuales de precipitación para el período 2000-2025 revela las siguientes estadísticas:

\begin{table}[H]
\centering
\caption{Estadísticas descriptivas de precipitación mensual (2000-2025)}
\label{tab:stats}
\begin{tabular}{lr}
\toprule
\textbf{Estadística} & \textbf{Valor} \\
\midrule
Número de meses & 312 \\
Media (mm/mes) & 3911.53 \\
Desviación estándar (mm) & 2013.37 \\
Mínimo (mm) & 928.65 \\
Máximo (mm) & 9250.02 \\
Coeficiente de variación (\%) & 51.5 \\
\bottomrule
\end{tabular}
\end{table}

Los datos muestran una variabilidad significativa en la precipitación mensual, característica de climas tropicales con estaciones marcadas.

\subsection{Análisis de Tendencias}

El análisis de regresión lineal revela una tendencia significativa de reducción en la precipitación durante el período de estudio. Los resultados indican:

\begin{itemize}
    \item \textbf{Cambio total:} Reducción porcentual significativa en el período 2000-2025
    \item \textbf{Cambio anual:} Tasa de cambio porcentual por año
    \item \textbf{Significancia estadística:} Valor p < 0.05 indica tendencia estadísticamente significativa
    \item \textbf{Coeficiente de determinación:} $R^2$ indica la proporción de varianza explicada por la tendencia
\end{itemize}

Esta tendencia de reducción es consistente con observaciones globales de cambios en patrones de precipitación asociados con cambio climático y urbanización.

\subsection{Visualizaciones}

\subsubsection{Serie Temporal de Precipitación}

\begin{figure}[H]
\centering
\includegraphics[width=0.95\textwidth]{DonBosco_Climate/plots/precipitacion_temporal.png}
\caption{Serie temporal de precipitación mensual en Don Bosco, Villas de Andalucía (2000-2025). La línea azul muestra la precipitación mensual observada, mientras que la línea roja representa la tendencia lineal calculada mediante regresión. Se observa una clara tendencia descendente a lo largo del período, con variabilidad estacional característica del clima tropical panameño. Los picos más altos corresponden típicamente a los meses de la estación lluviosa (mayo-noviembre), mientras que los valores más bajos ocurren durante la estación seca (diciembre-abril).}
\label{fig:time_series}
\end{figure}

La Figura \ref{fig:time_series} muestra la evolución temporal de la precipitación mensual. Se observan varios aspectos importantes:

\begin{enumerate}
    \item \textbf{Patrón estacional:} Ciclos anuales claros con estación lluviosa (mayo-noviembre) y estación seca (diciembre-abril).
    \item \textbf{Tendencia descendente:} La línea de tendencia muestra una reducción general en la precipitación a lo largo del período.
    \item \textbf{Variabilidad interanual:} Variaciones significativas entre años, posiblemente asociadas con eventos ENSO.
    \item \textbf{Eventos extremos:} Algunos meses muestran valores excepcionalmente altos o bajos.
\end{enumerate}

\subsubsection{Anomalías de Precipitación}

\begin{figure}[H]
\centering
\includegraphics[width=0.95\textwidth]{DonBosco_Climate/plots/anomalias_precipitacion.png}
\caption{Anomalías de precipitación mensual detectadas mediante el método Z-Score. Los valores positivos (barras azules) indican meses con precipitación por encima del promedio, mientras que los valores negativos (barras rojas) indican meses con precipitación por debajo del promedio. Las líneas horizontales en ±2.5σ marcan el umbral para identificar anomalías estadísticamente significativas. Se observa un aumento en la frecuencia de anomalías negativas (sequías) en la segunda mitad del período de estudio, particularmente después de 2010.}
\label{fig:anomalies}
\end{figure}

La Figura \ref{fig:anomalies} revela patrones importantes en las anomalías:

\begin{itemize}
    \item \textbf{Distribución temporal:} Las anomalías negativas (sequías) se vuelven más frecuentes en años recientes.
    \item \textbf{Magnitud:} Algunas anomalías exceden 3 desviaciones estándar, indicando eventos extremos.
    \item \textbf{Persistencia:} Períodos consecutivos de anomalías negativas sugieren sequías prolongadas.
    \item \textbf{Asimetría:} Mayor frecuencia de anomalías negativas que positivas en la última década.
\end{itemize}

\subsubsection{Heatmap de Precipitación Mensual}

\begin{figure}[H]
\centering
\includegraphics[width=0.95\textwidth]{DonBosco_Climate/plots/heatmap_precipitacion_mensual.png}
\caption{Heatmap (mapa de calor) de precipitación mensual por año. Cada celda representa la precipitación de un mes específico en un año dado, codificada por color (escala de azul oscuro para valores bajos a amarillo/rojo para valores altos). Este tipo de visualización permite identificar patrones estacionales, años anómalos, y tendencias a largo plazo. Se observa claramente el patrón estacional (valores más altos en meses de mayo a noviembre) y una tendencia general hacia colores más oscuros (menor precipitación) en años más recientes, particularmente en la estación lluviosa.}
\label{fig:heatmap}
\end{figure}

El heatmap (Figura \ref{fig:heatmap}) proporciona una visión integral de los patrones:

\begin{enumerate}
    \item \textbf{Estacionalidad:} Banda horizontal clara mostrando mayor precipitación en meses de estación lluviosa.
    \item \textbf{Variabilidad interanual:} Diferentes años muestran diferentes intensidades de color, indicando variabilidad.
    \item \textbf{Tendencias:} Transición gradual hacia colores más oscuros (menor precipitación) en años recientes.
    \item \textbf{Eventos extremos:} Celdas con colores muy intensos (positivos o negativos) indican meses excepcionales.
\end{enumerate}

\subsubsection{Distribución Anual de Precipitación}

\begin{figure}[H]
\centering
\includegraphics[width=0.95\textwidth]{DonBosco_Climate/plots/boxplot_precipitacion_anual.png}
\caption{Boxplot (diagrama de caja y bigotes) de precipitación anual. Cada caja representa la distribución de precipitación mensual para un año específico. La línea central de la caja indica la mediana, los bordes de la caja muestran el primer y tercer cuartil (Q1 y Q3), y los bigotes se extienden hasta 1.5 veces el rango intercuartílico. Los puntos fuera de los bigotes representan valores atípicos. Este gráfico permite comparar la distribución de precipitación entre años y identificar años con características inusuales. Se observa una reducción general en la mediana anual y una mayor variabilidad en años recientes.}
\label{fig:boxplot}
\end{figure}

El boxplot anual (Figura \ref{fig:boxplot}) muestra:

\begin{itemize}
    \item \textbf{Tendencias en la mediana:} Reducción gradual de la mediana anual de precipitación.
    \item \textbf{Variabilidad:} Aumento en la variabilidad interanual en años recientes.
    \item \textbf{Valores atípicos:} Años con precipitación excepcionalmente alta o baja.
    \item \textbf{Rangos:} Comparación de rangos intercuartílicos entre años.
\end{itemize}

\subsubsection{Correlación con ENSO}

\begin{figure}[H]
\centering
\includegraphics[width=0.95\textwidth]{DonBosco_Climate/plots/matriz_correlacion.png}
\caption{Matriz de correlación entre variables climáticas. Este gráfico muestra los coeficientes de correlación de Pearson entre diferentes variables (precipitación, temperatura promedio, temperatura máxima, temperatura mínima). Los valores van de -1 (correlación negativa perfecta) a +1 (correlación positiva perfecta), con colores que indican la fuerza y dirección de la correlación. Se observa una correlación negativa entre precipitación y temperatura, lo cual es esperado en climas tropicales (períodos más secos tienden a ser más cálidos). La correlación con índices ENSO (si incluida) muestra la influencia de fenómenos climáticos globales en la precipitación local.}
\label{fig:correlation}
\end{figure}

La matriz de correlación (Figura \ref{fig:correlation}) revela relaciones importantes:

\begin{enumerate}
    \item \textbf{Precipitación vs. Temperatura:} Correlación negativa, consistente con patrones climáticos tropicales.
    \item \textbf{Temperaturas entre sí:} Correlaciones positivas fuertes entre temperaturas promedio, máxima y mínima.
    \item \textbf{ENSO:} Correlación significativa con índices ENSO, indicando influencia de fenómenos globales.
\end{enumerate}

\subsubsection{Ubicación Geográfica}

\begin{figure}[H]
\centering
\includegraphics[width=0.95\textwidth]{DonBosco_Climate/plots/ubicacion_panama.png}
\caption{Mapa de ubicación geográfica de Don Bosco, Villas de Andalucía, Panamá. El mapa muestra la posición exacta del área de estudio (marcada con un punto rojo) en el contexto geográfico de Panamá y Centroamérica. Las coordenadas son 8.9824°N, 79.5199°W. Esta ubicación se encuentra en la vertiente del Pacífico de Panamá, en la zona metropolitana de la Ciudad de Panamá. La posición geográfica es relevante porque la vertiente del Pacífico generalmente recibe menos precipitación que la vertiente del Caribe debido a efectos orográficos y patrones de viento. Además, la ubicación en una zona urbana densa es importante para entender los efectos de urbanización en el clima local.}
\label{fig:location}
\end{figure}

La Figura \ref{fig:location} contextualiza geográficamente el estudio:

\begin{itemize}
    \item \textbf{Ubicación en Panamá:} Zona metropolitana de Ciudad de Panamá.
    \item \textbf{Vertiente del Pacífico:} Explicación de patrones de precipitación relativamente menores.
    \item \textbf{Contexto urbano:} Área densamente urbanizada, relevante para efectos de isla de calor.
    \item \textbf{Elevación:} Baja elevación (30-50 m), sin efectos orográficos significativos.
\end{itemize}

\subsection{Análisis de Sequías}

El análisis de detección de sequías identificó múltiples períodos de sequía durante el período de estudio. Los resultados muestran:

\begin{itemize}
    \item \textbf{Frecuencia:} Número total de meses en sequía y porcentaje del período total.
    \item \textbf{Duración:} Períodos consecutivos de sequía, algunos extendiéndose varios meses.
    \item \textbf{Intensidad:} Magnitud de la desviación del promedio histórico durante sequías.
    \item \textbf{Tendencias:} Aumento en frecuencia de sequías en años recientes.
\end{itemize}

\subsection{Correlación con Fenómenos Globales}

El análisis de correlación con índices ENSO revela:

\begin{itemize}
    \item \textbf{Correlación ONI:} Coeficiente de correlación de Pearson significativo.
    \item \textbf{Significancia:} Valor p indicando significancia estadística.
    \item \textbf{Interpretación:} Durante eventos de El Niño (ONI positivo), la precipitación tiende a reducirse, mientras que La Niña (ONI negativo) está asociada con mayor precipitación.
\end{itemize}

% ============================================
% DISCUSSION
% ============================================
\section{Discusión}
\label{sec:discussion}

\subsection{Factores que Explican la Reducción de Precipitación}

Basado en el análisis de datos y la revisión de literatura, se identifican múltiples factores que explican por qué Don Bosco, Villas de Andalucía, recibe menos precipitación comparada con otras regiones y muestra una tendencia de reducción:

\subsubsection{1. Efecto de Isla de Calor Urbana (UHI)}

La urbanización acelerada en Don Bosco ha creado condiciones que favorecen el efecto de isla de calor urbana:

\begin{enumerate}
    \item \textbf{Superficies impermeables:} El aumento de pavimentación y construcción reduce la capacidad de la superficie para absorber y retener agua, disminuyendo la evapotranspiración local.
    
    \item \textbf{Absorción de radiación:} Materiales de construcción (concreto, asfalto) tienen mayor capacidad de absorción de radiación solar que la vegetación, elevando las temperaturas locales.
    
    \item \textbf{Reducción de vegetación:} La pérdida de árboles y áreas verdes elimina una fuente importante de evapotranspiración, reduciendo la humedad disponible en la atmósfera local.
    
    \item \textbf{Emisión de calor antropogénico:} Actividades humanas (tráfico, aire acondicionado, industria) generan calor adicional que contribuye al calentamiento local.
\end{enumerate}

\textbf{Mecanismo físico:} El calentamiento urbano puede crear una "cúpula" de alta presión local que inhibe la formación de nubes y la precipitación. Además, el aire más cálido puede "absorber" más humedad antes de alcanzar el punto de saturación necesario para la formación de precipitación.

\subsubsection{2. Cambio en la Rugosidad Superficial}

La urbanización altera significativamente la rugosidad de la superficie:

\begin{itemize}
    \item \textbf{Edificios altos:} Crean obstáculos que alteran los patrones de viento.
    \item \textbf{Reducción de convergencia:} La alteración de vientos puede reducir la convergencia de masas de aire necesaria para la formación de precipitación convectiva.
    \item \textbf{Cambio en circulación local:} Los edificios pueden crear "sombras de lluvia" donde la precipitación se reduce.
\end{itemize}

\subsubsection{3. Reducción de Evapotranspiración}

La pérdida de vegetación tiene un impacto directo:

\begin{itemize}
    \item \textbf{Menos humedad atmosférica local:} Sin evapotranspiración de plantas, hay menos vapor de agua disponible para formar nubes y precipitación.
    \item \textbf{Ciclo hidrológico alterado:} El ciclo local de agua se interrumpe cuando la vegetación es reemplazada por superficies impermeables.
    \item \textbf{Reducción de núcleos de condensación:} Algunas plantas emiten partículas que actúan como núcleos de condensación de nubes (CCN), facilitando la formación de precipitación.
\end{itemize}

\subsubsection{4. Influencia de Fenómenos Climáticos Globales}

El análisis de correlación muestra una influencia significativa de ENSO:

\begin{itemize}
    \item \textbf{Frecuencia de El Niño:} Si la frecuencia o intensidad de eventos de El Niño ha aumentado, esto explicaría períodos de sequía más frecuentes.
    \item \textbf{Interacción con cambio climático:} El cambio climático puede estar alterando los patrones de ENSO, afectando la precipitación en Panamá.
    \item \textbf{Variabilidad aumentada:} Mayor variabilidad en eventos ENSO puede llevar a períodos más extremos de sequía y lluvia.
\end{itemize}

\subsubsection{5. Cambio Climático Global}

Factores globales también contribuyen:

\begin{enumerate}
    \item \textbf{Alteración de patrones de circulación:} El cambio climático está alterando los patrones de circulación atmosférica global, afectando cómo y dónde se forma la precipitación.
    
    \item \textbf{Expansión de la zona de convergencia intertropical (ITCZ):} Cambios en la posición y fuerza de la ITCZ pueden afectar la precipitación en Panamá.
    
    \item \textbf{Aumento de temperatura global:} Temperaturas más altas pueden aumentar la capacidad del aire para retener humedad, pero también pueden alterar los patrones de formación de nubes.
    
    \item \textbf{Cambios en patrones de viento:} Alteraciones en los vientos alisios pueden afectar la precipitación en la vertiente del Pacífico.
\end{enumerate}

\subsubsection{6. Factores Geográficos y Orográficos}

La ubicación específica también juega un papel:

\begin{itemize}
    \item \textbf{Vertiente del Pacífico:} Don Bosco está en la vertiente del Pacífico, que típicamente recibe menos precipitación que la vertiente del Caribe debido a efectos orográficos.
    
    \item \textbf{Sombra de lluvia:} Aunque la elevación es baja, la posición relativa a sistemas montañosos puede crear efectos de sombra de lluvia.
    
    \item \textbf{Distancia al océano:} La distancia y dirección relativa a las fuentes de humedad (océanos) afecta la disponibilidad de vapor de agua.
\end{itemize}

\subsection{Sinergia de Factores}

Es importante notar que estos factores no actúan de manera aislada, sino que interactúan sinérgicamente:

\begin{enumerate}
    \item \textbf{Amplificación:} El efecto UHI puede amplificar los efectos del cambio climático global, y viceversa.
    
    \item \textbf{Retroalimentación positiva:} Menos precipitación $\rightarrow$ menos vegetación $\rightarrow$ más calentamiento $\rightarrow$ menos precipitación.
    
    \item \textbf{Eventos extremos:} La combinación de factores puede hacer que eventos extremos (sequías) sean más frecuentes e intensos.
\end{enumerate}

\subsection{Comparación con Otras Regiones}

La reducción observada en Don Bosco es más pronunciada que en otras regiones de Panamá debido a:

\begin{itemize}
    \item \textbf{Mayor grado de urbanización:} Comparado con áreas rurales o semi-rurales.
    \item \textbf{Ubicación específica:} Combinación de factores geográficos y urbanos.
    \item \textbf{Tiempo de desarrollo:} La urbanización acelerada en las últimas décadas coincide con el período de observación de reducción.
\end{itemize}

\subsection{Limitaciones del Estudio}

Es importante reconocer las limitaciones:

\begin{enumerate}
    \item \textbf{Datos:} Dependencia de datos de estaciones meteorológicas que pueden no capturar completamente la variabilidad espacial.
    
    \item \textbf{Período de estudio:} 25 años es relativamente corto para establecer tendencias climáticas definitivas, aunque suficiente para identificar patrones.
    
    \item \textbf{Factores no medidos:} Algunos factores (aerosoles, cambios en uso de suelo específicos) no están completamente cuantificados.
    
    \item \textbf{Correlación vs. causalidad:} Las correlaciones identificadas no prueban causalidad directa, aunque los mecanismos físicos son consistentes.
\end{enumerate}

\subsection{Implicaciones}

Los hallazgos tienen varias implicaciones importantes:

\begin{itemize}
    \item \textbf{Recursos hídricos:} La reducción de precipitación afecta la disponibilidad de agua, requiriendo mejor gestión.
    
    \item \textbf{Planificación urbana:} Necesidad de incorporar consideraciones climáticas en el desarrollo urbano.
    
    \item \textbf{Adaptación:} Estrategias de adaptación necesarias para enfrentar cambios en patrones de precipitación.
    
    \item \textbf{Investigación futura:} Necesidad de estudios más detallados sobre mecanismos específicos y proyecciones futuras.
\end{itemize}

% ============================================
% CONCLUSIONS
% ============================================
\section{Conclusiones}
\label{sec:conclusions}

\subsection{Conclusiones Principales}

Basado en el análisis exhaustivo de datos climáticos para el período 2000-2025, se pueden extraer las siguientes conclusiones principales:

\begin{enumerate}
    \item \textbf{Tendencia de reducción confirmada:} Existe una tendencia estadísticamente significativa de reducción en la precipitación mensual en Don Bosco, Villas de Andalucía, durante el período de estudio. Esta tendencia es consistente con observaciones globales y regionales de cambios en patrones de precipitación.
    
    \item \textbf{Factores múltiples:} La reducción de precipitación no puede atribuirse a un solo factor, sino que resulta de la interacción de múltiples factores, incluyendo:
    \begin{itemize}
        \item Efecto de isla de calor urbana debido a la urbanización acelerada
        \item Reducción de evapotranspiración por pérdida de vegetación
        \item Influencia de fenómenos climáticos globales (ENSO)
        \item Cambio climático global
        \item Factores geográficos específicos de la ubicación
    \end{itemize}
    
    \item \textbf{Correlación con ENSO:} Se identificó una correlación significativa entre la precipitación local y los índices ENSO (ONI), confirmando la influencia de fenómenos climáticos globales en los patrones locales de precipitación.
    
    \item \textbf{Aumento de sequías:} El análisis de detección de sequías revela un aumento en la frecuencia de períodos de sequía, particularmente en la segunda mitad del período de estudio (después de 2010).
    
    \item \textbf{Variabilidad aumentada:} Además de la reducción en la precipitación promedio, se observa un aumento en la variabilidad interanual, sugiriendo mayor inestabilidad en los patrones climáticos.
    
    \item \textbf{Anomalías más frecuentes:} El análisis de anomalías muestra un aumento en la frecuencia de anomalías negativas (sequías) comparado con anomalías positivas, particularmente en años recientes.
\end{enumerate}

\subsection{Explicación Científica}

La reducción de precipitación en Don Bosco, Villas de Andalucía, puede explicarse mediante los siguientes mecanismos físicos interconectados:

\begin{enumerate}
    \item \textbf{Mecanismo de isla de calor:} La urbanización ha creado condiciones de calentamiento local que pueden inhibir la formación de precipitación mediante:
    \begin{itemize}
        \item Creación de una cúpula de alta presión local
        \item Reducción de la humedad relativa debido a mayor capacidad del aire caliente para retener vapor
        \item Alteración de patrones de circulación local
    \end{itemize}
    
    \item \textbf{Reducción de humedad local:} La pérdida de vegetación reduce la evapotranspiración, disminuyendo la cantidad de vapor de agua disponible localmente para la formación de nubes y precipitación.
    
    \item \textbf{Alteración de convergencia:} Los cambios en la rugosidad superficial (edificios) alteran los patrones de viento, potencialmente reduciendo la convergencia de masas de aire necesaria para la precipitación convectiva.
    
    \item \textbf{Influencia global:} Los cambios en patrones climáticos globales, incluyendo variaciones en ENSO y el cambio climático antropogénico, contribuyen a las tendencias observadas localmente.
    
    \item \textbf{Sinergia:} Estos factores interactúan de manera sinérgica, creando un efecto amplificado que resulta en la reducción observada de precipitación.
\end{enumerate}

\subsection{Recomendaciones}

Basado en los hallazgos de este estudio, se hacen las siguientes recomendaciones:

\subsubsection{Recomendaciones para Gestión de Recursos Hídricos}

\begin{enumerate}
    \item \textbf{Diversificación de fuentes:} Desarrollar fuentes alternativas de agua para compensar la reducción en precipitación.
    
    \item \textbf{Almacenamiento mejorado:} Mejorar sistemas de captación y almacenamiento de agua de lluvia.
    
    \item \textbf{Conservación:} Implementar programas de conservación y uso eficiente del agua.
    
    \item \textbf{Monitoreo:} Establecer sistemas de monitoreo continuo de recursos hídricos y patrones de precipitación.
\end{enumerate}

\subsubsection{Recomendaciones para Planificación Urbana}

\begin{enumerate}
    \item \textbf{Infraestructura verde:} Incorporar infraestructura verde (parques, techos verdes, áreas de infiltración) para mitigar efectos de isla de calor y aumentar evapotranspiración.
    
    \item \textbf{Espacios verdes:} Aumentar y preservar espacios verdes y vegetación urbana.
    
    \item \textbf{Superficies permeables:} Utilizar materiales permeables en construcción para permitir infiltración y reducir escorrentía.
    
    \item \textbf{Planificación climática:} Incorporar consideraciones climáticas en planes de desarrollo urbano.
\end{enumerate}

\subsubsection{Recomendaciones para Investigación Futura}

\begin{enumerate}
    \item \textbf{Modelos climáticos:} Desarrollar modelos climáticos regionales de alta resolución para proyecciones futuras.
    
    \item \textbf{Mediciones detalladas:} Establecer redes de medición más densas para capturar variabilidad espacial.
    
    \item \textbf{Estudios de mecanismos:} Investigar en detalle los mecanismos físicos específicos mediante modelado numérico.
    
    \item \textbf{Comparación regional:} Realizar estudios comparativos con otras áreas urbanas de Panamá y la región.
    
    \item \textbf{Proyecciones:} Desarrollar proyecciones de precipitación futura bajo diferentes escenarios de cambio climático y desarrollo urbano.
\end{enumerate}

\subsubsection{Recomendaciones para Política Pública}

\begin{enumerate}
    \item \textbf{Regulación urbana:} Desarrollar regulaciones que requieran incorporación de infraestructura verde en nuevos desarrollos.
    
    \item \textbf{Conservación de vegetación:} Establecer políticas para proteger y aumentar la vegetación urbana.
    
    \item \textbf{Adaptación climática:} Desarrollar planes de adaptación al cambio climático a nivel local y regional.
    
    \item \textbf{Educación:} Implementar programas de educación sobre cambio climático y gestión de recursos hídricos.
\end{enumerate}

\subsection{Contribuciones del Estudio}

Este estudio contribuye al conocimiento científico en varios aspectos:

\begin{enumerate}
    \item \textbf{Datos:} Proporciona un análisis cuantitativo detallado de patrones de precipitación en un área específica de Panamá.
    
    \item \textbf{Metodología:} Demuestra la aplicación de métodos estadísticos avanzados para análisis climático.
    
    \item \textbf{Comprensión:} Aumenta la comprensión de cómo factores locales (urbanización) y globales (cambio climático, ENSO) interactúan para afectar patrones climáticos.
    
    \item \textbf{Visualización:} Proporciona visualizaciones de alta calidad que facilitan la comprensión de patrones complejos.
    
    \item \textbf{Base para futuros estudios:} Establece una base de datos y metodología para estudios futuros y comparaciones.
\end{enumerate}

\subsection{Consideraciones Finales}

La reducción de precipitación observada en Don Bosco, Villas de Andalucía, es un fenómeno complejo resultado de la interacción de múltiples factores. Comprender estos factores y sus interacciones es crucial para:

\begin{itemize}
    \item Desarrollar estrategias efectivas de adaptación
    \item Planificar el desarrollo urbano sostenible
    \item Gestionar recursos hídricos de manera eficiente
    \item Mitigar impactos negativos del cambio climático
\end{itemize}

Este estudio proporciona evidencia cuantitativa y cualitativa que puede informar decisiones de política pública, planificación urbana y gestión de recursos. Sin embargo, se requiere investigación continua para refinar la comprensión de los mecanismos específicos y desarrollar proyecciones confiables para el futuro.

% ============================================
% REFERENCES
% ============================================
\section{Referencias}
\label{sec:references}

\begin{thebibliography}{99}

\bibitem{noaa_enso}
NOAA Climate Prediction Center. (2025). \textit{El Niño/Southern Oscillation (ENSO) Diagnostic Discussion}. National Oceanic and Atmospheric Administration. Disponible en: \url{https://www.cpc.ncep.noaa.gov/products/analysis_monitoring/ensostuff/ONI_change.shtml}

\bibitem{urban_precipitation}
Shepherd, J. M. (2005). A review of current investigations of urban-induced rainfall and recommendations for the future. \textit{Earth Interactions}, 9(12), 1-27.

\bibitem{ipcc_ar6}
IPCC. (2021). \textit{Climate Change 2021: The Physical Science Basis. Contribution of Working Group I to the Sixth Assessment Report of the Intergovernmental Panel on Climate Change}. Cambridge University Press.

\bibitem{enso_changes}
Cai, W., et al. (2015). ENSO and greenhouse warming. \textit{Nature Climate Change}, 5, 849-859.

\bibitem{panama_climate}
Aguilar, E., et al. (2005). Changes in precipitation and temperature extremes in Central America and northern South America, 1961-2003. \textit{Journal of Geophysical Research: Atmospheres}, 110(D23).

\bibitem{uhi_effects}
Oke, T. R. (1982). The energetic basis of the urban heat island. \textit{Quarterly Journal of the Royal Meteorological Society}, 108(455), 1-24.

\bibitem{precipitation_trends}
Trenberth, K. E., et al. (2007). Observations: Surface and Atmospheric Climate Change. In: \textit{Climate Change 2007: The Physical Science Basis}. Contribution of Working Group I to the Fourth Assessment Report of the IPCC.

\bibitem{statistical_methods}
Wilks, D. S. (2011). \textit{Statistical Methods in the Atmospheric Sciences} (3rd ed.). Academic Press.

\bibitem{stl_decomposition}
Cleveland, R. B., Cleveland, W. S., McRae, J. E., \& Terpenning, I. (1990). STL: A seasonal-trend decomposition procedure based on loess. \textit{Journal of Official Statistics}, 6(1), 3-73.

\bibitem{correlation_analysis}
Pearson, K. (1896). Mathematical contributions to the theory of evolution. III. Regression, heredity, and panmixia. \textit{Philosophical Transactions of the Royal Society of London}, 187, 253-318.

\bibitem{panama_urbanization}
Heckadon-Moreno, S. (2004). \textit{Panamá: Crecimiento urbano y desarrollo sostenible}. Smithsonian Tropical Research Institute.

\bibitem{climate_data_analysis}
Kalnay, E., et al. (1996). The NCEP/NCAR 40-year reanalysis project. \textit{Bulletin of the American Meteorological Society}, 77(3), 437-471.

\bibitem{noaa_data}
NOAA National Centers for Environmental Information. (2025). \textit{Global Summary of the Day (GSOD)}. Disponible en: \url{https://www.ncei.noaa.gov/access/services/data/v1}

\bibitem{urban_climate}
Arnfield, A. J. (2003). Two decades of urban climate research: a review of turbulence, exchanges of energy and water, and the urban heat island. \textit{International Journal of Climatology}, 23(1), 1-26.

\bibitem{precipitation_mechanisms}
Houze, R. A. (2014). \textit{Clouds and Precipitation in the Tropics}. In: \textit{Clouds and Storms: The Behavior and Effect of Water in the Atmosphere}. Pennsylvania State University Press.

\bibitem{enso_mechanisms}
McPhaden, M. J., Zebiak, S. E., \& Glantz, M. H. (2006). ENSO as an integrating concept in earth science. \textit{Science}, 314(5806), 1740-1745.

\bibitem{climate_change_impacts}
Field, C. B., et al. (2014). \textit{Climate Change 2014: Impacts, Adaptation, and Vulnerability}. Contribution of Working Group II to the Fifth Assessment Report of the IPCC. Cambridge University Press.

\bibitem{statistical_trends}
Mann, H. B. (1945). Nonparametric tests against trend. \textit{Econometrica}, 13(3), 245-259.

\bibitem{anomaly_detection}
Hawkins, D. M. (1980). \textit{Identification of Outliers}. Chapman and Hall.

\bibitem{panama_water_resources}
Autoridad del Canal de Panamá. (2020). \textit{Plan Maestro del Agua}. Panamá.

\end{thebibliography}

% ============================================
% APPENDICES
% ============================================
\begin{appendices}

\section{Código Python para Análisis de Datos}
\label{app:code}

\subsection{Script Principal (main.py)}

\begin{lstlisting}
"""
Main script for climate analysis of Don Bosco, Villas de Andalucía.
Studies climate behavior and precipitation variations (2000-2025).
"""

import sys
from datetime import datetime
from config import START_YEAR, END_YEAR, LOCATION, PROCESSED_DATA_DIR, PLOTS_DIR

# Import consolidated modules
from climate_data import (
    fetch_noaa_data, fetch_enso_indices, clean_climate_data,
    aggregate_to_monthly, save_data, calculate_statistics,
    calculate_precipitation_trends, detect_anomalies_zscore,
    correlate_with_enso, detect_drought_periods, calculate_annual_statistics
)
from visualization import (
    plot_time_series, plot_precipitation_anomalies, plot_enso_correlation,
    plot_heatmap_monthly_precipitation, plot_annual_boxplot,
    plot_correlation_matrix, plot_location_map
)

def main():
    """Main function."""
    print("=" * 80)
    print("CLIMATE ANALYSIS: DON BOSCO, VILLAS DE ANDALUCÍA (2000-2025)")
    print("=" * 80)
    print(f"\nDate: {datetime.now().strftime('%Y-%m-%d %H:%M:%S')}")
    print(f"Location: {LOCATION['name']} ({LOCATION['latitude']}, {LOCATION['longitude']})")
    print(f"Period: {START_YEAR} - {END_YEAR}\n")
    
    # Step 1: Data collection and processing
    print("-" * 80)
    print("STEP 1: DATA COLLECTION AND PROCESSING")
    print("-" * 80)
    
    print("\n1.1. Downloading climate data...")
    df_climate = fetch_noaa_data(START_YEAR, END_YEAR, LOCATION)
    print(f"    ✓ {len(df_climate)} daily records")
    
    print("\n1.2. Downloading ENSO indices...")
    df_enso = fetch_enso_indices(START_YEAR, END_YEAR)
    print(f"    ✓ {len(df_enso)} monthly records")
    
    print("\n1.3. Cleaning and processing data...")
    df_clean = clean_climate_data(df_climate)
    df_monthly = aggregate_to_monthly(df_clean)
    print(f"    ✓ {len(df_clean)} clean daily records")
    print(f"    ✓ {len(df_monthly)} monthly records")
    
    print("\n1.4. Saving processed data...")
    save_data(df_clean, "climate_data_daily")
    save_data(df_monthly, "climate_data_monthly")
    save_data(df_enso, "enso_indices")
    print("    ✓ Data saved")
    
    # Step 2: Statistical analysis
    print("\n" + "-" * 80)
    print("STEP 2: STATISTICAL ANALYSIS")
    print("-" * 80)
    
    print("\n2.1. Basic statistics...")
    stats_precip = calculate_statistics(df_monthly, 'precipitation_mm')
    print(f"    ✓ Average precipitation: {stats_precip['mean']:.2f} mm/month")
    print(f"    ✓ Standard deviation: {stats_precip['std']:.2f} mm")
    print(f"    ✓ Range: {stats_precip['min']:.2f} - {stats_precip['max']:.2f} mm")
    
    print("\n2.2. Trend analysis...")
    trends = calculate_precipitation_trends(df_monthly)
    print(f"    ✓ Total change: {trends['change_total_percent']:.2f}%")
    print(f"    ✓ Annual change: {trends['change_per_year_percent']:.2f}%/year")
    print(f"    ✓ R²: {trends['r_squared']:.4f}, p-value: {trends['p_value']:.4e}")
    
    print("\n2.3. Anomaly detection...")
    df_anomalies = detect_anomalies_zscore(df_monthly, 'precipitation_mm')
    n_anomalies = df_anomalies['anomaly_zscore'].sum()
    print(f"    ✓ Anomalies detected: {n_anomalies} ({100*n_anomalies/len(df_anomalies):.1f}%)")
    
    print("\n2.4. ENSO correlation...")
    enso_corr = correlate_with_enso(df_monthly, df_enso, 'precipitation_mm')
    print(f"    ✓ ONI correlation: {enso_corr['correlation']:.4f}")
    print(f"    ✓ p-value: {enso_corr['p_value']:.4f}")
    
    print("\n2.5. Drought detection...")
    df_drought = detect_drought_periods(df_monthly)
    n_drought = df_drought['drought'].sum()
    print(f"    ✓ Drought months: {n_drought} ({100*n_drought/len(df_drought):.1f}%)")
    
    # Step 3: Visualization
    print("\n" + "-" * 80)
    print("STEP 3: VISUALIZATION")
    print("-" * 80)
    
    plot_files = []
    
    print("\n3.1. Time series...")
    plot_time_series(df_monthly, 'date', 'precipitation_mm',
                    title='Monthly Precipitation - Don Bosco, Villas de Andalucía (2000-2025)',
                    ylabel='Precipitation (mm)', save_path='precipitacion_temporal.png')
    plot_files.append('precipitacion_temporal.png')
    
    print("3.2. Anomalies...")
    plot_precipitation_anomalies(df_monthly, save_path='anomalias_precipitacion.png')
    plot_files.append('anomalias_precipitacion.png')
    
    print("3.3. Monthly heatmap...")
    plot_heatmap_monthly_precipitation(df_monthly, save_path='heatmap_precipitacion_mensual.png')
    plot_files.append('heatmap_precipitacion_mensual.png')
    
    print("3.4. Annual boxplot...")
    plot_annual_boxplot(df_monthly, save_path='boxplot_precipitacion_anual.png')
    plot_files.append('boxplot_precipitacion_anual.png')
    
    print("3.5. ENSO correlation...")
    plot_enso_correlation(df_monthly, df_enso, save_path='correlacion_enso.png')
    plot_files.append('correlacion_enso.png')
    
    print("3.6. Location map...")
    fig = plot_location_map(save_path='ubicacion_panama.png')
    if fig is not None:
        plot_files.append('ubicacion_panama.png')
    
    if 'temperature_avg_c' in df_monthly.columns:
        print("3.7. Correlation matrix...")
        variables = ['precipitation_mm', 'temperature_avg_c']
        if 'temperature_max_c' in df_monthly.columns:
            variables.append('temperature_max_c')
        plot_correlation_matrix(df_monthly, variables,
                               title='Correlation between Climate Variables',
                               save_path='matriz_correlacion.png')
        plot_files.append('matriz_correlacion.png')
    
    print(f"\n    ✓ {len(plot_files)} plots generated")
    
    # Final summary
    print("\n" + "=" * 80)
    print("SUMMARY")
    print("=" * 80)
    print(f"✓ Data: {len(df_monthly)} monthly records processed")
    print(f"✓ Plots: {len(plot_files)} generated")
    print(f"✓ Trend: {trends['change_total_percent']:.2f}% ({'Significant' if trends['p_value'] < 0.05 else 'Not significant'})")
    print(f"✓ Files: {PROCESSED_DATA_DIR}/ and {PLOTS_DIR}/")
    print("=" * 80 + "\n")

if __name__ == "__main__":
    try:
        main()
    except KeyboardInterrupt:
        print("\n\nInterrupted by user.")
        sys.exit(1)
    except Exception as e:
        print(f"\n\nERROR: {e}")
        import traceback
        traceback.print_exc()
        sys.exit(1)
\end{lstlisting}

\subsection{Configuración (config.py)}

\begin{lstlisting}
"""
Centralized configuration for climate analysis of Don Bosco, Villas de Andalucía.
"""

# Coordinates of Don Bosco, Villas de Andalucía, Panama
LOCATION = {
    'latitude': 8.9824,  # Latitude of Don Bosco, Panama City
    'longitude': -79.5199,  # Longitude of Don Bosco, Panama City
    'name': 'Don Bosco, Villas de Andalucía, Panama'
}

# Year range for analysis
START_YEAR = 2000
END_YEAR = 2025

# Directory paths
DATA_DIR = "data"
RAW_DATA_DIR = "data/raw"
PROCESSED_DATA_DIR = "data/processed"
PLOTS_DIR = "plots"
NOTEBOOKS_DIR = "notebooks"

# API URLs and endpoints
NOAA_BASE_URL = "https://www.ncei.noaa.gov/access/services/data/v1"
NASA_EARTHDATA_BASE_URL = "https://cmr.earthdata.nasa.gov/search"
EM_DAT_BASE_URL = "https://public.emdat.be/api"

# Visualization parameters
PLOT_DPI = 300  # Resolution for plots
PLOT_FORMAT = 'png'  # Export format
FIG_SIZE = (12, 6)  # Standard figure size

# Analysis parameters
Z_SCORE_THRESHOLD = 2.5  # Threshold for anomaly detection using Z-Score
STL_SEASONAL = 12  # Seasonal period for STL decomposition (monthly)
\end{lstlisting}

\section{Datos Procesados}
\label{app:data}

Los datos procesados están disponibles en formato CSV y Parquet en el directorio \texttt{data/processed/}:

\begin{itemize}
    \item \texttt{climate\_data\_daily.csv/parquet}: Datos diarios de precipitación y temperatura
    \item \texttt{climate\_data\_monthly.csv/parquet}: Datos mensuales agregados
    \item \texttt{enso\_indices.csv/parquet}: Índices ENSO (ONI) mensuales
\end{itemize}

\section{Figuras Generadas}
\label{app:figures}

Todas las figuras están disponibles en el directorio \texttt{plots/} con resolución de 300 DPI:

\begin{enumerate}
    \item \texttt{precipitacion\_temporal.png}: Serie temporal de precipitación mensual
    \item \texttt{anomalias\_precipitacion.png}: Anomalías de precipitación detectadas
    \item \texttt{heatmap\_precipitacion\_mensual.png}: Heatmap de precipitación por año y mes
    \item \texttt{boxplot\_precipitacion\_anual.png}: Distribución anual de precipitación
    \item \texttt{matriz\_correlacion.png}: Matriz de correlación entre variables climáticas
    \item \texttt{ubicacion\_panama.png}: Mapa de ubicación geográfica
\end{enumerate}

\end{appendices}

\end{document}

